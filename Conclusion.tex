\subsection{Team A}
The general consensus for this module was that there was poor error checking/handling. While most use cases work well and perform desired tasks, there are many instances where there might be an error that the user will not know about. This could result in the system failing or being non-responsive without the user having an explanation as to why this is the case. What little error checking did occur, they were console logged instead of thrown as an exception, again resulting in potential system failure without reporting the reason to the user.
\newline \newline
Most software (non-functional) requirements were met, although reliability could be compromised because of the poor error handling/checking. Auditabilty isn't possible because of the absense of a tracking table in the database which was however beyond the scope of this module. 
\newline \newline
Technologies used were in line with the specification, with a few (mocha, unit.js electrolyte) not being used at all.

\subsection{Team B}
The biggest issue in this module were the assumptions made about data being used in operations. Simply using data and not testing validy could result in garbage values being used in operations and undesired output being produced. There were also some components of certain use cases that failed, making the use case not work as expected. This is obviously not ideal.
\newline \newline
Maintainability is compromised by the un-readablity of the code, as well as code duplication in functions. Adding functionality would be difficult.  

\subsection{Overall}
Overall, team A have created a better reporting module than team B. Team B have issues which could result in bigger errors than team A, and with Team B's module compromising the Maintainability requirement, it makes it the less desirable choice to fix for the bigger system. 
\newline \newline
Both modules need work however as neither is ready to be put in as part of a greater system. In a purely black and white sense they both failed to meet the full requirements of the specification. However it is recommended that Team A be the one chosen should there be an attempt to modify a module to use for the final system. 