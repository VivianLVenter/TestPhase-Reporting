\subsection{Functional Requirements Review}
%Tienie
\subsubsection{Threads.getThreadStats}








%Vivian
\subsubsection{Import Thread Appraisal}
\paragraph{Data in an external file that was created using the exportThreadAppraisal function is used.\\} 
\textbf{\emph{Partial Passed}} \\
The module do make use of a file. The file is sent as arguments in the importThreadAppraisal function (parameters are directory and  fileName). So in this function they assume this is the file created by the exportThreadAppraisal function.
\par

\paragraph{The data set is associated with only one member and only one specified appraisal.\\}
\textbf{\emph{Failed}} \\
The function never actually check if the data set is about only one member and only one specific appraisal. Therefore the data cannot be deemed as eligible for import.
\par

\paragraph{A record contains all detail about the post along with a field that should contain an ordinal number that represents the levels of the specified appraisal.\\}
\textbf{\emph{Passed}} \\
The record does contain all detail about the post and a field that represents the leves of the specified appraisal which is the appraisalValue.
\par

\paragraph{Edits to the data is ignored when importing a thread appraisal.\\}
\textbf{\emph{Failed}} \\
The function does not prevent edits to the data and there is not clear indicated of ignoring such edits.
\par

\paragraph{For each record the assignAppraisalToPost function is applied.\\}
\textbf{\emph{Passed}} \\
The function does call assignAppraisalToPost for each record.
\par

\paragraph{The appraisal level as stored in the file for each post is updated as an appraisal assigned by the member associated with the data set.\\}
\textbf{\emph{Failed}} \\
The function does not use the appraisal level to update the data set.
\par

\paragraph{Check validity of member and appraisal.\\}
\textbf{\emph{Partial Passed}} \\
The function does check the validity of the member, however they have a dummy function that only returns true.\\
The function does check the validity of the appraisal, however no exception is thrown/raised it only returns true or false and there is also no means of catching an exception, that is the isValid function is not surrounded with try/catch blocks to catch exceptions and therefore the service delivery is not stopped when the appraisal is not valid.\\
The appraisal level is check for out of range, however no exception is thrown/raised when is it out of range and there the service delivery is not stopped if the appraisal level is out of range.
\par


\subsubsection{}



\subsubsection{}


